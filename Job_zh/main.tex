\documentclass[11pt,a4paper]{moderncv}

\moderncvtheme[blue]{classic} 
\usepackage[utf8]{inputenc}  %Windows 
\usepackage{ctex}

%\usepackage[scale=0.975]{geometry}
\usepackage[top=0.5cm, bottom=0.5cm, left=0.5cm, right=0.5cm]{geometry}
\usepackage{graphicx}

\firstname{赵明星}
\familyname{}
\title{Python,C++,Java}   
\mobile{(+86) 188 1086 0130}                       
\email{zhaomingxingDL@gmail.com}    
\homepage{igoingdown.github.io}
\social[github]{igoingdown}

\photo[50pt]{photo.png}     
\makeatletter
\renewcommand*{\bibliographyitemlabel}{\@biblabel{\arabic{enumiv}}}
\makeatother
\quote{\textbf{NLP研发工程师}} 
\usepackage{multibib}
\newcites{book,misc}{{Books},{Others}}

\nopagenumbers{}                         
\begin{document}
\maketitle
\section{教育经历}
\cventry{2016--2019}{北京邮电大学}{网络技术研究院, 网络与交换技术国家重点实验室}{硕士研究生}{GPA 79/100}{}
\cventry{2012--2016}{北京邮电大学}{计算机学院}{本科}{GPA 76/100,  \textbf{CET6: 536}}{}

\section{获奖经历}
\cvlistdoubleitem{“编程之美”2017\textbf{亚军},2/1118}{全国研究生数学建模竞赛\textbf{二等奖}}
\cvlistdoubleitem{百度\&西交大大数据竞赛优秀奖,\textbf{10/1393}}{HackPKU 2017\textbf{车道线检测赛题冠军}}
\cvlistdoubleitem{“挑战杯”北京市大学生课外作品竞赛\textbf{二等奖}}{校级优秀研究生干部,\textbf{top 1\%}}

\section{实习经历}
\cventry{2018.05 -- 2018.07}{微软亚洲研究院} {研究实习生}{}{}
{\begin{itemize}
\item 在真实的Excel数据集上,设计并实现\textbf{SeqGAN}回收表中有可能被用于生成chart的列的组合,在recall为82.4\%时数据集中正样本数量增加65\%。
%\item 在SeqGAN中使用基于\textbf{Monte Carlo搜索}的方法代替仅仅使用最终生成的组合与用户用于绘制chart使用的列的组合的相似度作为强化学习中的Q-function,提升了action的估值准确率。
\item 在真实的Power BI数据集上,基于TF-IDF算法构造加权词向量特征,设计并实现\textbf{类VGG}模型预测用户可能生成的chart的类型。precision比加入RNN模型高13\%,比最好的ML模型(RF)模型高出8\%。%也实现了ResNet,但是很容易过拟合
\end{itemize}}

\cventry{2017.02 --  2017.06}{时速云}{后台研发实习生}{}{}
{
\begin{itemize}
\item 使用\textbf{distributed tensorflow}实现在公有云平台上\textbf{多机多卡并行}的\textbf{Vgg模型},收敛速度比单机版提速 \textbf{80\%}。
\item \textbf{独立开发新版本的公有云后台管理系统的前端和后端},在原有系统中,添加对\textbf{docker集群}状态的实时监控。使用Golang、\textbf{Beego}框架开发基于\textbf{kubernets}管理docker集群的后台服务,使用\textbf{Angular JS}进行前端开发。
\end{itemize}}

\section{项目经历}
\cventry{2016.10 -- 2018.12}{基于深度学习的知识库问答研究}{}{}{}
{
\begin{itemize}
\item 单关系问答中,使用\textbf{tensorflow}实现\textbf{seq2seq}的模型,加入\textbf{多粒度表达}模块提升特征的表示能力,加入\textbf{copy机制}动态调整源端词用于生成答案的概率,并加入\textbf{attention},accuracy比纯seq2seq模型提升14.6\%。%继续挖掘!!!罗列工具,而不是insight。!!!钻研能力!!!!,不是学习工具。
\item 多关系问答中,使用\textbf{pytorch}实现\textbf{强化学习}模型解决了\textbf{强监督信号缺失}的问题。使用迭代学习学到伪最佳答案,作为强化学习模型的弱监督信号,解决了强化学习启动阶段\textbf{reward稀疏}的问题,模型收敛速度提升350\%。
%\item 使用\textbf{Spring Boot}框架实现API server,java服务直接与\textbf{virtuoso} 交互,\textbf{sparql}查询的速度比之前基于解析html的方法提升\textbf{200\%}。
\end{itemize}}

\cventry{2017.05 -- 2017.08}{“编程之美”2017挑战赛}{亚军}{}{}
{
\begin{itemize}
\item 资格赛使用pytorch实现融入\textbf{attention}和\textbf{copy机制}的seq2seq模型,\textbf{MRR > 66\%},\textbf{排名13/1118}。
\item 初赛使用C\#语言, 利用luis和\textbf{bot framework}设计并实现能够回答北邮相关问题的问答机器人,\textbf{排名3/200}。
\item 决赛阶段为初赛设计的问答机器人添加了“找工作”、“找对象”等实用功能,\textbf{决赛排名2/8}。
\item 在bot开发阶段,我负责\textbf{全部后台开发},主要包括单轮和多轮对话的设计、对话状态转移、信息存储和检索等。使用python实现爬取北邮人论坛的\textbf{实时爬虫},\textbf{基于 TF-IDF 实现搜索引擎}。
\end{itemize}}

\section{科研经历}
\cventry{Dec. 2017}{参与实验室项目组隐私保护的关键研究}{}{}{在项目组中负责实现图和序列上的\textbf{差分隐私算法}的实现,在多个真实数据集上进行实验,在隐私约束下数据可用性达到业界领先水平并与导师合作发表论文}{X. Cheng, S. Su, S. Xu, L. Xiong, K. Xiao and \textbf{M. Zhao}. A Two-Phase Algorithm for Differentially Private Frequent Subgraph Mining. IEEE \textbf{Transactions on Knowledge \& Data Engineering}, 30(8):1411-1425, 2018}
\end{document}
